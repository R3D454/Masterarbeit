\chapter{Einleitung}\label{Einleitung}
\onehalfspacing
In den letzten Jahrzehnten spielten Simulatoren in zivilen und militärischen Bereichen eine immer größere Rolle. Durch sie konnte die Ausbildung von Piloten aber auch von normalen Fahrzeugbesatzungen und Soldaten verbessert werden. Durch den Einsatz von Simulationen können Szenarien in sicherer Umgebung geübt werden. Neben der Ausbildung von Personal können in Simulationen auch Fahrzeuge oder Geräte getestet werden, ohne diese konstruieren zu müssen. 
Um die Ausbildung oder das Simulieren von Fahrzeugen noch effizienter zu gestalten, kam die Forderung auf, einzelne Simulationen miteinander zu vernetzten und interagieren zu lassen. Das Vernetzen von Simulation integriert zum Beispiel Flugsimulatoren und Panzersimulatoren in ein Szenario und lässt diese zusammen interagieren. 
Eine Möglichkeit um Simulationen miteinander zu vernetzen, ist \ac{dis}.\\
In dieser Masterarbeit werden zunächst die Grundsätze der \ac{oop} erklärt. Weiterhin wird eine Programmiersprache beschrieben, mithilfe der objektorientiert programmiert werden kann. Anschließend werden  Inhalte des \ac{dis} Standards, welche in dieser Masterarbeit verwendet wurden. Neben den genannten Grundlagen werden Lösungsansätze erklärt und verglichen, die für die Erstellung einer Klassenhierarchie infrage kommen. Des weiteren wird der Lösungsansatz und die Implementierung  erklärt, die gewählt wurde. Im Anschluss daran werden die aktuellen Funktionalitäten erklärt und es werden mögliche Erweiterung dargestellt. Abschließend folgt ein Fazit und ein Ausblick.  
       