\chapter{Einleitung}\label{Einleitung}
\onehalfspacing
In den letzten Jahrzehnten spielten Simulatoren in zivilen und militärischen Bereichen eine immer größere Rolle. Durch sie konnte die Ausbildung von Piloten aber auch von normalen Fahrzeugbesatzungen und Soldaten verbessert werden. Durch den Einsatz von Simulationen können Szenarien in sicherer Umgebung geübt werden. Neben der Ausbildung von Personal können in Simulationen auch Fahrzeuge oder Geräte getestet werden, ohne diese konstruieren zu müssen. 
Um die Ausbildung oder das Simulieren von Fahrzeugen noch effizienter zu gestalten, kam die Forderung auf, einzelne Simulationen miteinander zu vernetzten und interagieren zu lassen. Das Vernetzen von Simulation integriert zum Beispiel Flugsimulatoren und Panzersimulatoren in ein Szenario und lässt diese zusammen interagieren. 
Eine Möglichkeit um Simulationen miteinander zu vernetzen, ist \ac{dis}.
\\ In dieser Studienarbeit werden erste Erfahrungen mit \ac{dis} gemacht, und es wird eine Einschätzung getroffen, ob eine open soucre Library die nötige Funktionalität  liefert, um eine Simulation zu erstellen. Dabei wird zu Beginn  der Arbeit ein Überblick über den \ac{dis} Standard gegeben. Außerdem werden die wichtigsten \ac{pdu} des \ac{ieee} Standard erklärt.    Anschließend wird eine Beispielsimulation beschrieben, die im Rahmen der Erprobung der Library erstellt wurde. Hierbei wird besonders auf die Erstellung einer \acl{espdu} und auf die gelösten Probleme eingegangen.
Abschließend wird eine Einschätzung getroffen, ob die verwendete Library die nötigen Funktionen enthält oder ob es nötig ist weitere, Libarys einzubinden oder eigene Funktionen zu erstellen.        